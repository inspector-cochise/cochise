
\documentclass{article}

\usepackage[utf8]{inputenc}
\usepackage[ngerman]{babel}
\usepackage[T1]{fontenc}
\usepackage[osf]{mathpazo}
\linespread{1.05}

\usepackage[a4paper,left=30mm,right=30mm,top=35mm,,bottom=35mm]{geometry}
\usepackage[colorlinks=false]{hyperref}
\usepackage{url}

\title{Inspector-Cochise\footnote{Siehe \url{www.inspector-cochise.de}} \ Benutzerhandbuch}
\author{von akquinet\footnote{Siehe \url{www.akquinet.de}, Kontakt: \texttt{Immanuel.Sims$\texttt{@}$akquinet.de}}}

\newenvironment{code}{\begin{quote}\tt}{\end{quote}}

\begin{document}

\maketitle

%\tableofcontents

\section{Über Inspector-Cochise}
\emph{Inspector-Cochise} oder einfach nur \emph{Cochise} ist ein Tool, welches dabei helfen soll das BSI\footnote{\emph{Bundesamt für
Sicherheit in der Informationstechnik}}-Audit für den Apache Web-Server\footnote{
\url{https://www.bsi.bund.de/cae/servlet/contentblob/478398/publicationFile/30917/isi_web_server_checkliste_apache_pdf.pdf}}
durchzuführen. Dabei werden lästige und fehleranfällige Aufgaben, wie das Überprüfen von Dateirechten, sowie die Gültigkeit bestimmter
Konfigurationsmerkmale automatisiert durchgeführt. Bei komplexeren Aufgaben werden Hilfen zur Seite gestellt um diese so komfortabel
wie möglich auszuführen.
\par
Cochise ist nach einem Häuptling der Chokonen-Apachen benannt um den Bezug zum \emph{Apache} Web-Server herzustellen. 

\section{Systemvorraussetzungen}
Um Cochise erfolgreich auszuführen wird relativ wenig benötigt:
\begin{itemize}
  \item grundlegende POSIX Tools (wie \texttt{sh}, \texttt{grep}, \texttt{awk}, \texttt{find}, \ldots)
  \item Java (1.6 oder höher)
  \item \texttt{root}-Rechte
  \item \emph{optional:} ein Terminal mit viel Cache (zum Hochscrollen) 
\end{itemize}
Falls kein entsprechendes Java installiert ist kann wie im Folgenden beschrieben eine temporäre Java Installation eingerichtet werden, welche
keine Einfluss auf das restliche System hat und einfach wieder gelöscht werden kann.

\subsection*{Temporäre Java Installation}
Auf der offiziellen Java-Download Seite\footnote{\url{http://www.java.com/de/download/manual.jsp?locale=de}} kann man eine selbstextrahierende
Datei (mit der Endung \texttt{.bin}!) herunterladen. Diese kann man in ein beliebiges Verzeichnis entpacken.
\begin{code}
	cd ihr/verzeichnis/\\
	sh nameDerHeruntergeladenenDatei.bin
\end{code}
Dadurch wir in \texttt{ihr/verzeichnis} ein Verzeichnis mit einem Name in etwa wie \texttt{jre1.6.0\_26} erstellt. Darin befindet sich die
temporäre Java-Installation. Nun müssen Sie nur noch Ihre \texttt{PATH}-Variable entsprechend anpassen (und u.~U. vorher sichern) und dann
ist Ihr System bzgl. Java bereit Cochise auszuführen.
\begin{code}
	echo \$PATH > PATH\_bu\\
	PATH=ihr/verzeichnis/jre1.6.0\_26/bin:\$PATH
\end{code}
In der gerade eben benutzten Shell werden nun die Java-Kommandos der temporären Installation genutzt. Wenn Sie wie in obigem Code die
\texttt{PATH}-Variable gesichert haben, können Sie diese wie folgt wieder zurücksetzen:
\begin{code}
	PATH=`cat PATH\_bu`
\end{code}
(die \texttt{`} sind Backticks)

\section{Installation, Aufruf und Bedienung}
Cochise wird als \texttt{.tar.gz}-Archiv ausgeliefert. Entpacken Sie dieses Archiv in ein Verzeichnis Ihrer Wahl.
\begin{code}
	cd ihr/verzeichnis/\\
	gzip -dc cochise.tar.gz | tar -xv
\end{code}
\par
Cochise wird nun einfach über ein Skript gestartet, wozu Sie allerding \texttt{root}-Rechte benötigen. Cochise führt keine Änderungen
an Ihrer Konfiguration durch, sondern macht lediglich Vorschläge. Die \texttt{root}-Rechte werden zum lesen einiger Daten benötigt.
\begin{code}
	./startAudit.sh
\end{code}
Cochise meldet sich daraufhin mit einer Willkommensbotschaft (die Sie zumindest einmal lesen sollten) und beginnt den relevanten Teil Ihrer
Systemkonfiguration und Anforderungen an den Einsatz des Apache Web-Server abzufragen.
\par
Bei den meisten Fragen ist in eckigen Klammern ein Standardwert vorgegeben, wenn Sie einfach \textsf{\small$\langle$Enter$\rangle$} drücken,
wird dieser Wert akzeptiert. Falls dieser Wert nicht passt, können Sie natürlich immer einen anderen Eingeben.
\begin{code}
	Wie ist Ihr Name? \lbrack Cochise\rbrack \emph{ \textsf{\small$\langle$Enter$\rangle$}}\\
	Hallo Cochise.\\
	Woher kommen Sie? \lbrack Indien\rbrack \emph{ Nordamerika \textsf{\small$\langle$Enter$\rangle$}}\\
	Oh, Nordamerika, dort soll es schön sein.
\end{code}
\par
Leider ist im Dokument des BSI-Audits keine Nummerierung der Fragen angegeben. Cochise verwendet zur besseren Übersicht eine Nummerierung,
die sich ergibt indem man von 1 beginnend die Hauptfragen durchnummeriert. Unterfragen werden von \textsf{a} beginnend durchnummeriert,
erhalten also eine Nummer wie z.~B. \textsf{5a} oder \textsf{9b}.

\section{Nützliche Features}

In der allerersten Zeile der Ausgabe teil Cochise Ihnen den Namen einer \texttt{.html}-Datei mit, in der das Ergebnis des Audits besser
lesbar als auf der Konsole abgespeichert wird. Dazu muss Cochise allerdings ganz bis zum Ende Ausgeführt werden. Dort finden Sie auch einige
Erklärungen und Beispiele, die Auf der Konsole zu viel Platz einnehmen würden.

\end{document}