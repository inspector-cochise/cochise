
\documentclass{article}

\usepackage[utf8]{inputenc}
\usepackage[english]{babel}
\usepackage[T1]{fontenc}
\usepackage[osf]{mathpazo}
\linespread{1.05}

\usepackage[a4paper,left=30mm,right=30mm,top=35mm,,bottom=35mm]{geometry}
\usepackage[colorlinks=false]{hyperref}
\usepackage{url}

\title{Inspector-Cochise\footnote{See \url{www.inspector-cochise.com}} \ User's Manual}
\author{by akquinet\footnote{See \url{www.akquinet.de}, contact: \texttt{Immanuel.Sims$\texttt{@}$akquinet.de}}}

\newenvironment{code}{\begin{quote}\tt}{\end{quote}}

\begin{document}

\maketitle

%\tableofcontents

\section{About Inspector-Cochise}
\emph{Inspector-Cochise} or just \emph{Cochise} is a tool which will help you to perform the BSI\footnote{\emph{Bundesamt für Sicherheit
in der Informationstechnik} translated from the german it would be called \emph{Federal Bureau for Security in Information-Technology}}-audit
for the apache web-server\footnote{
\url{https://www.bsi.bund.de/cae/servlet/contentblob/478398/publicationFile/30917/isi_web_server_checkliste_apache_pdf.pdf}}.
It performs some tedious and error-prone tasks like checking of file-permissions or the validity of some configuration-feature fully automatic.
For complexer tasks it offers help to make it more comfortable and less error-prone to perform this task.
\par
Cochise is named after an indian chief of the Chokonen Apache to emphasize the connection to the apache web-server.

\section{System-Requirements}
Cochise has only few system-requirements:
\begin{itemize}
  \item some basic POSIX-tools (like \texttt{sh}, \texttt{grep}, \texttt{awk}, \texttt{find}, \ldots)
  \item Java (1.6 or higher)
  \item \texttt{root}-access to your system
  \item \emph{optional:} a terminal with much cache (to scroll up)
\end{itemize}
If you haven't installed a proper java on your system you can follow the guide below to create a temporary java installation which only
exists in a special directory you choose and which can easily be deleted cleanly.

\subsection*{A Temporary Java Installation}
On the official Java-download-page\footnote{\url{http://www.java.com/en/download/manual.jsp?locale=en}} you can download a self-extracting
file (with name-ending \texttt{.bin}!). You can unpack this in any directory you want.
\begin{code}
	cd your/directory/\\
	sh nameOfTheDownloadedFile.bin
\end{code}
This will create a directory in \texttt{your/directory} with a name like \texttt{jre1.6.0\_26}. In that directory is your temporary java-
installation. Now you only need to adapt your \texttt{PATH}-varaible (and maybe store it somewhere befor) and your system is ready for cochise
(talking about java).
\begin{code}
	echo \$PATH > PATH\_bu\\
	PATH=your/directory/jre1.6.0\_26/bin:\$PATH
\end{code}
The new java-commands will now be used from the shell you used to perform the last command. If you stored your initial \texttt{PATH}-variable
like in the above code you can restore it as follows:
\begin{code}
	PATH=`cat PATH\_bu`
\end{code}
(the \texttt{`} are backticks)

\section{Installation, Call and Usage}
Cochise is being deployed as \texttt{.tar.gz}-archive. Extract this archive to a directory you choose.
\begin{code}
	cd your/directory/\\
	gzip -dc cochise.tar.gz | tar -xv
\end{code}
\par
To start Cochise you just need to start a script, but you will need to do this as \texttt{root}. Cochise won't change any of your system's
configuration, it will only make propositions. \texttt{root}-access is needed to read some data (which only \texttt{root} should be able to
access). 
\begin{code}
	./startAudit.sh
\end{code}
Cochise will now show up with a welcome message (which you should read at least one time) and then starts to retrieve some information
about your systems' configuration and your requirements on the apahe web-server.
\par
In the most cases Cochise asks a question, it provides a default-answer in square-bracktets. If that default-answer fits you just need
to press \textsf{\small$\langle$Enter$\rangle$}, if not you can enter your own value.
\begin{code}
	What's your name? \lbrack Cochise\rbrack \emph{ \textsf{\small$\langle$Enter$\rangle$}}\\
	Hello Cochise.\\
	Where do you come from? \lbrack India\rbrack \emph{ North-America \textsf{\small$\langle$Enter$\rangle$}}\\
	Oh, North-America, it's nice there.
\end{code}
\par
Unfortunately in the paper presenting the BSI-audit no numbers for the questions are provided. For a better overview Cochise uses it's own
numbering. Major questions are numbered beginning from 1. Subquestions are numbered from \textsf{a}, i.~e. they get numbers like \textsf{5a}
or \textsf{9b}.

\section{Useful Features}

In the first line of it's output Cochise tells you the name of an \texttt{.html}-file which will contain a report of your audit. You will
need to run Cochise until it's end to get this report. It is not only better readable, but it also contains additional explenations and
examples which would be too much for the command-line.


\end{document}